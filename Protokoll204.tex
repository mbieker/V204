\documentclass[11pt]{article}
%Gummi|061|=)
\usepackage{amsmath}
\usepackage{amsthm}
\usepackage{amsbsy}
\usepackage{amssymb}
\usepackage{inputenc}
\usepackage{graphicx}
\usepackage{selinput}
\SelectInputMappings{%
adieresis={ä},
germandbls={ß},
}
\title{\textbf{Versuch V204: Wärmeleitung von Metallen}}
\author{Martin Bieker\\
		Julian Surmann\\
		\\
		Durchgef\"{u}hrt am 31.10.2013\\
		Tu Dortmund}
\date{}
\usepackage{graphicx}
\begin{document}
\renewcommand\tablename{Tabelle}
\renewcommand\figurename{Abbildung}
\maketitle
\thispagestyle{empty}
\newpage
\clearpage
\setcounter{page}{1}


\section{Einleitung}
Im folgenden Versuch geht es um die Untersuchung von Wärmeleitung. Unter Wärmeleitung versteht man den Fluss von Wärme in Richtung geringerer Temparatur. Es wird die Wärmeleitung von mehreren Metallen untersucht. 
\section{Theorie}
Für die Untersuchung des Wärmetransportes in Metallen wird hier nur die Wärmeleitung betrachtet, Konvektion und Wärmestrahlung sind vernachlässigbar.
Die Wärmemenge, die in einer Zeit dt durch die Querschnittsfläche A des zu untersuchenden Stabes fließt, ist gegeben durch:
\begin{equation}
dQ = -\kappa A \frac{\partial T}{\partial x} dt.
\end{equation}
Die Wärmestromdichte $j_w$ ist dann gegeben mit
\begin{equation}
j_w = -\kappa \frac{\partial T}{\partial x}.
\end{equation}
Aus Formel (2) und der Kontinuitätsbedingung lässt sich eine eindimensionale Wärmeleitungsgleichung herleiten:
\begin{equation}
\frac{\partial T}{\partial t} = \frac{\kappa}{\rho c} \frac{\partial^2 T}{\partial x^2}.
\end{equation}
Diese Gleichung gibt die zeitliche und räumliche Entwicklung der Temparaturverteilung an. $c$ steht hier für die spezifische Wärme des Metalles, $\rho$ für dessen Dichte.
$\sigma_T = \frac{\kappa}{\rho c}$ steht damit für die Temparaturleitfähigkeit des Materials. Die Temparaturleitfähigkeit ist ein Maß für Geschwindigkeit, mit der Temparaturdifferenzen neutralisiert werden.
Im zweiten Versuchsteil soll das Ende verschiedener Metallstäbe periodisch erwärmt und gekühlt werden. Dabei entsteht eine räumliche und zeitliche Temparaturwelle. Diese hat die Form
\begin{equation}
T(x,t) = T_{max} e^{\sqrt{\frac{w \rho c}{2 \kappa}}x}cos \left( wt- \sqrt{\frac{w \rho c}{2 \kappa}}x \right).
\end{equation}
Die Phasengeschwindigkeit der Welle ist gegeben mit
\begin{equation}
v = \frac{w}{k} = \frac{w}{\sqrt{\frac{w \rho c}{2 \kappa}}} = \sqrt{\frac{2 \kappa w}{\rho c}}
\end{equation}
Mithilfe der Dämpfung, die aus dem Verhältnis $\frac{A_{nah}}{A_fern}$ folgt, und der Ausdrücke $w = \frac{2 \pi}{T^*}$ ($T^*$ steht für die Periodendauer, nicht für eine Temparatur) und $\Phi = 2 \pi \frac{\Delta t}{T^*}$ folgt für die Wärmeleitfähigkeit
\begin{equation}
\kappa = \frac{\rho c (\Delta x)^2}{2 \Delta t ln(\frac{A_{nah}}{A_{fern}})}
\end{equation}
Hierbei beschreibt $\Delta x$ den Abstand zwischen den beiden Temparaturmessstellen und $\Delta t$ die Phasendifferenz der Wärmewelle zwischen den beiden Messstellen.
\section{Aufbau und Durchf\"{u}hrung}
Zur Messung der Temparaturen der Metallstäbe ist ein Versuchsaufbau auf einer Grundplatte befestigt. Auf dieser Grundplatte sind vier Metallstäbe angebracht, an denen je zwei Thermoelemente in einem festgelegten Abstand befestigt sind. Ein Peltier-Element kann über seine Unterseite das eine Ende der Metallstäbe simultan erhitzen oder kühlen. Über ein Labornetzteil wird das Peltierelement mit Spannung versorgt, durch die Möglichkeit einer Umpolung kann das Peltier-Element entweder kühlen oder heizen. Die Temparaturen an den acht Thermoelementen werden von einem sogenannten Temparaturarray ausgelesen und per Datenkabel an einen Xplorer GLX \footnote{Kleincomputer zur Datenerfassung und Auswertung mit Druckfunktion für Tabellen und Graphen} gesendet.
\subsection{Eigenschaften der Metallstäbe}
Über die viel Metallstäbe sind viele Eigenschaften bekannt. Sie werden in der folgenden Tabelle dargestellt:
\begin{table}[h]
\centering
\begin{tabular}{|c|c|c|c|c|}
\hline
Stoff & Abmessungen [cm] & $\rho \left[ \frac{kg}{m^3} \right] $ & $M [N]$ & $ D [\frac{Nm}{rad}]$ \\
\hline
20.0 & 0.349 & 0.1 & 0.01 & 0.0286\\
40.0 & 0.698 & 0.2 & 0.02 & 0.0286\\
\hline
\end{tabular}
\caption{Eigenschaften der Metallstäbe}
\end{table}




\section{Auswertung}
Hier folgt die Auswertung.
Das ist eine geeignete Tabelle mit 5 Spalten:
\begin{table}[h]
\centering
\begin{tabular}{|c|c|c|c|c|}
\hline
$\varphi[^\circ]$ & $\varphi [rad]$ & $F [N]$ & $M [N]$ & $ D [\frac{Nm}{rad}]$ \\
\hline
20.0 & 0.349 & 0.1 & 0.01 & 0.0286\\
40.0 & 0.698 & 0.2 & 0.02 & 0.0286\\
\hline
\end{tabular}
\caption{Benennung der Tabelle}
\end{table}
\section{Diskussion}
Hier kommt die Diskussion hin.
\section{Literatur- und Abbildungsverzeichnis}
Hier befindet sich das Literatur- und Abbildungsverzeichnis.
\section{Anhang}
Hier stehen die im Anhang angefügten Dokumente.
\end{document}
